\documentclass{article}

% if you need to pass options to natbib, use, e.g.:
% \PassOptionsToPackage{numbers, compress}{natbib}
% before loading nips_2017
%
% to avoid loading the natbib package, add option nonatbib:
% \usepackage[nonatbib]{nips_2017}

\usepackage{nips_2017}

% to compile a camera-ready version, add the [final] option, e.g.:
% \usepackage[final]{nips_2017}

\usepackage[utf8]{inputenc} % allow utf-8 input
\usepackage[T1]{fontenc}    % use 8-bit T1 fonts
\usepackage{hyperref}       % hyperlinks
\usepackage{url}            % simple URL typesetting
\usepackage{booktabs}       % professional-quality tables
\usepackage{amsfonts}       % blackboard math symbols
\usepackage{nicefrac}       % compact symbols for 1/2, etc.
\usepackage{microtype}      % microtypography

\title{Associating gene expression to neural network derived image features}

% The \author macro works with any number of authors. There are two
% commands used to separate the names and addresses of multiple
% authors: \And and \AND.
%
% Using \And between authors leaves it to LaTeX to determine where to
% break the lines. Using \AND forces a line break at that point. So,
% if LaTeX puts 3 of 4 authors names on the first line, and the last
% on the second line, try using \AND instead of \And before the third
% author name.

\author{
  David S.~Hippocampus\thanks{Use footnote for providing further
    information about author (webpage, alternative
    address)---\emph{not} for acknowledging funding agencies.} \\
  Department of Computer Science\\
  Cranberry-Lemon University\\
  Pittsburgh, PA 15213 \\
  \texttt{hippo@cs.cranberry-lemon.edu  asdfas} \\
  %% examples of more authors
  %% \And
  %% Coauthor \\
  %% Affiliation \\
  %% Address \\
  %% \texttt{email} \\
  %% \AND
  %% Coauthor \\
  %% Affiliation \\
  %% Address \\
  %% \texttt{email} \\
  %% \And
  %% Coauthor \\
  %% Affiliation \\
  %% Address \\
  %% \texttt{email} \\
  %% \And
  %% Coauthor \\
  %% Affiliation \\
  %% Address \\
  %% \texttt{email} \\
}

\begin{document}
% \nipsfinalcopy is no longer used

\maketitle

\begin{abstract}
I will be updating this document with the progress of  my NIPS 2017 submission.
\end{abstract}

\section*{Plan}

\subsection*{Q1: How much variability is shared between image features and expression?} 
\begin{enumerate}
\item Calculate PCA of image features and expression

\item Calculate cross correlations between the PCs until 99.9\% variance is explained.

\item Use the cross correlation measurements to derive total fraction of variance explained of images by expression: calculate $\sum_i \lambda_i \sum_j r^2_{ij} $ where $\lambda_i$ is the $i$th eigenvalue of the image feature PCA, $r^2_{ij}$ is the correlation of image feature PC $i$ with expression feature PC $j$.

\item Make all of the above one-line-to-run for any combination of patch size, feature layer, tissue

\item Report the value for each patch size, feature layer, tissue; interpret findings

\end{enumerate}

\subsection*{Q2: How much variability is shared between image features and expression?} 

\begin{enumerate}
\item Regress out known confounders $C$ from the image feature $F$ and expression data E. I.e., fit models $F \sim C + \epsilon$ and $E \sim C + \epsilon$. Use log scale for Ischemic time, monitor all scatter plots for confounders that are included in the model that the linear model assumptions hold. Decide whether to retain full model of all confounders x all genes/features, retain only significant associations, or do forward feature selection.

\item Repeat steps Q1:1-5 above using the residuals of this model. 

\item Create a scatter plot of total shared variance vs. technical shared variance [again, all tissues etc.]

\item Interpret findings in context of many tissues, scales, feature layers
\end{enumerate}

\subsection*{Q3: What genes share expression variability to visual features, and at which scales?}
\begin{enumerate}
\item Using residuals of technical confounders, fit linear model of expression vs. image feature - calculate effect sizes and p-values

\item Apply multiple testing correction to derive a list of image feature associated genes.

\item Sort genes based on variance explained by each image feature, perform GSEA or gProfiler ranked gene list analysis to test whether particular aspects of biology are enriched. Interpret

\item Assess number of associations [total and per-feature] for different tissues, feature scales. Assess overlap of findings from gene list analyses. Interpret.
\end{enumerate}

\subsection*{Q3A: What genes are influenced by technical confounders?}
\begin{enumerate}

\item For each confounder, calculate the fraction of variance it explains for each gene expression level. 

\item Sort genes based on variance explained by confounder, perform GSEA or gProfiler ranked gene list analysis to test whether particular aspects of biology are enriched.]

\end{enumerate}

\subsection*{Once these done}
\begin{itemize}
\item Q4: What do the image features represent?
\item Q5: Is there a genetic basis to the image features?
\item Q6: Are there gene expression levels that cause image feature changes, or vice versa?
\item Q7: Are there image features that are associated with clinical annotations? Are any causal?

\end{itemize}






%[1] Alexander, J.A.\ \& Mozer, M.C.\ (1995) Template-based algorithms
%for connectionist rule extraction. In G.\ Tesauro, D.S.\ Touretzky and
%T.K.\ Leen (eds.), {\it Advances in Neural Information Processing
%  Systems 7}, pp.\ 609--616. Cambridge, MA: MIT Press.
%
%[2] Bower, J.M.\ \& Beeman, D.\ (1995) {\it The Book of GENESIS:
%  Exploring Realistic Neural Models with the GEneral NEural SImulation
%  System.}  New York: TELOS/Springer--Verlag.
%
%[3] Hasselmo, M.E., Schnell, E.\ \& Barkai, E.\ (1995) Dynamics of
%learning and recall at excitatory recurrent synapses and cholinergic
%modulation in rat hippocampal region CA3. {\it Journal of
%  Neuroscience} {\bf 15}(7):5249-5262.

\end{document}
