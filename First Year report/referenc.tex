%%%%%%%%%%%%%%%%%%%%%%%% referenc.tex %%%%%%%%%%%%%%%%%%%%%%%%%%%%%%
% sample references
% %
% Use this file as a template for your own input.
%
%%%%%%%%%%%%%%%%%%%%%%%% Springer-Verlag %%%%%%%%%%%%%%%%%%%%%%%%%%
%


\begin{thebibliography}{99.}

\bibitem{image-net} Krizhevsky, A., Sutskever, I., & Hinton, G. E. (2012). ImageNet Classification with Deep Convolutional Neural Networks. Advances in Neural …, 1097–1105.

\bibitem{complex-sources-of-variation} McCall, M. N., Illei, P. B., & Halushka, M. K. (2016). Complex Sources of Variation in Tissue Expression Data: Analysis of the GTEx Lung Transcriptome. The American Journal of Human Genetics, 99(3), 624–635. http://doi.org/10.1016/j.ajhg.2016.07.007

\bibitem{scalable-unsupervised-learning} Lee, H., Grosse, R., Ranganath, R., & Ng, A. Y. (2009). Convolutional deep belief networks for scalable unsupervised learning of hierarchical representations. Proceedings of the 26th annual … (pp. 609–616). ACM. http://doi.org/10.1145/1553374.1553453

\bibitem{deep-inside-convolutional-networks} Simonyan, K., Vedaldi, A., & Zisserman, A. (2013, December 20). Deep Inside Convolutional Networks: Visualising Image Classification Models and Saliency Maps. arXiv.org.

\bibitem{science-journal} Gando, G., Yamada, T., Sato, H., Oyama, S., & Kurihara, M. (2016). Fine-tuning deep convolutional neural networks for distinguishing illustrations from photographs. Expert Systems with Applications, 66, 295–301. http://doi.org/10.1016/j.eswa.2016.08.057

\bibitem{science-journal} Lin, M., Chen, Q., & Yan, S. (2013, December 16). Network In Network. arXiv.org.
\end{thebibliography}







